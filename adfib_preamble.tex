\usepackage{amsfonts, amsthm, amssymb, amsmath}
\usepackage{mathtools}
\usepackage{graphicx,caption,subcaption}
\usepackage{comment}
\usepackage{xcolor}
\usepackage{tikz}
%\usetikzlibrary{decorations.markings,arrows.meta,calc,fit,quotes,cd,math,external}
\usetikzlibrary{
  matrix,
  arrows,
  arrows.meta,
  calc,
  fit,
  quotes,
  cd,
  math,
  external,
  shapes,
  decorations.markings,
  decorations.pathreplacing,
  patterns,
  decorations.pathmorphing
}

\usepackage{url}
\usepackage[inline]{enumitem}
	\setlist{itemsep=0em, topsep=0em, parsep=0em}
	\setlist[enumerate]{label=(\alph*)}
\usepackage[all,2cell]{xy}\UseAllTwocells\SilentMatrices
\usepackage[breaklinks=true]{hyperref}%John's choices, can change; previous choice \definecolor{hyperrefcolor}{rgb}{0,0,0.7}
\definecolor{darkgreen}{rgb}{0,0.45,0}
\definecolor{myurlcolor}{rgb}{0.6,0,0}
\definecolor{mycitecolor}{rgb}{0,0,0.8}
\definecolor{myrefcolor}{rgb}{0,0,0.8}
\hypersetup{colorlinks, linkcolor=myrefcolor, citecolor=darkgreen, urlcolor=myurlcolor,
final,linktoc=page}
\usepackage[capitalize]{cleveref}
\crefname{equation}{}{}
\crefname{item}{}{}
\crefname{prop}{Proposition}{Propositions}

%
% environments and counters
%

\newtheorem{thm}{Theorem}[section]
\newtheorem{cnj}[thm]{Conjecture}
\newtheorem{lem}[thm]{Lemma}
\newtheorem{prop}[thm]{Proposition}
\newtheorem{cor}[thm]{Corollary}

\theoremstyle{remark}
	\newtheorem{remark}[thm]{Remark}
	\newtheorem{notation}[thm]{Notation}

\theoremstyle{definition}
	\newtheorem{ex}[thm]{Example} 
	\newtheorem{defn}[thm]{Definition}

% math text formatting
\newcommand{\cat}[1]{\mathsf{#1}}

% common category names

\newcommand{\Set}{\cat{Set}}
\newcommand{\Grph}{\cat{Grph}}
\newcommand{\Cat}{\cat{Cat}}
\newcommand{\twoCat}{\cat{2Cat}}
\newcommand{\Adj}{\cat{Adj}}
\newcommand{\one}{\mathbf{1}}
\newcommand{\two}{\mathbf{2}}
\newcommand{\Fib}{\cat{Fib}}
\newcommand{\OpFib}{\cat{OpFib}}
\newcommand{\Corefl}{\cat{Corefl}}
\newcommand{\Rali}{\cat{Rali}}

\newcommand{\C}{\cat{A}}
\newcommand{\D}{\cat{X}}
\newcommand{\A}{\cat{A}}
\newcommand{\X}{\cat{X}}
\newcommand{\U}{U}
\newcommand{\Q}{Q}
\renewcommand{\L}{L}
\newcommand{\R}{R}
\renewcommand{\P}{P}
\newcommand{\B}{\cat{B}}
\newcommand{\Y}{\cat{Y}}

\newcommand{\Cocart}{\mathrm{Cocart}}
\newcommand{\Cart}{\mathrm{Cart}}

\newcommand{\op}[1]{\operatorname{#1}}
\renewcommand{\t}[1]{\text{#1}}

\newcommand{\from}{\colon}
\newcommand{\xto}[1]{\xrightarrow{#1}}
\newcommand{\To}{\Rightarrow}
\newcommand{\Tto}{\Rrightarrow}
\newcommand{\bydef}{\coloneqq}
\newcommand{\define}{\textbf}%consider change to bold? or something else?

%
% math operators
%

\DeclareMathOperator{\Hom}{Hom}
\DeclareMathOperator{\id}{id}
\DeclareMathOperator{\ob}{Ob}
\DeclareMathOperator{\arr}{arr}
\DeclareMathOperator{\im}{im}
\DeclareMathOperator{\Aut}{Aut}
\DeclareMathOperator{\Bij}{Bij}
\DeclareMathOperator{\Sub}{Sub}
\DeclareMathOperator{\colim}{colim}
\newcommand{\iso}{\cong}

%
% tikz styles
%

% arrows for commuting diagram
\tikzset{
  cd/.style={
    ->,
    scale=6,
    >=angle 90,
    font=\scriptsize}
  }

% its us!

\definecolor{purple(x11)}{rgb}{0.5, 0.0, 0.5}
\newcommand{\chris}{\color{purple(x11)}}
\newcommand{\daniel}{\color{red}}
